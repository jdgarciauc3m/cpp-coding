\section{I: Interfaces}

\begin{frame}[t]{Interfaces Guidelines}
\begin{itemize}
  \item Interface as contract between two parts of a prorgram.
  \item Good interfaces essential aspect of code organization.

  \mode<presentation>{\vfill\pause}
  \item Guidelines:
  \begin{itemize}
    \item I.1: Make interfaces explicit
    \item I.2: Avoid non-const global variables
    \item I.3: Avoid singletons
    \item I.4: Make interfaces precisely and strongly typed
    \item I.5: State preconditions (if any)
    \item I.6: Prefer Expects() for expressing preconditions
    \item I.7: State postconditions
    \item I.8: Prefer Ensures() for expressing postconditions
    \item I.9: If an interface is a template, document its parameters using concepts
    \item I.10: Use exceptions to signal a failure to perform a required task
  \end{itemize}
\end{itemize}
\end{frame}

\subsection{I1: Make interfaces explicit}

\begin{frame}[t,fragile]{Explicit interfaces}
\begin{itemize}
  \item Assumptions that are not part of the interface are:
    \begin{itemize}
      \item Hard to test.
      \item Very easily forgotten.
    \end{itemize}
\begin{block}{Bad}
\begin{lstlisting}
int round_value(double x) {
  if (rounding == round_mode::up) { return std::ceil(x); }
  else { return std::floor(x); }
}
\end{lstlisting}
\end{block}

\mode<presentation>{\vfill\pause}
\begin{block}{Good: avoid invisible dependencies}
\begin{lstlisting}
int round_value(double x, round_mode rounding) {
  if (rounding == round_mode::up) { return std::ceil(x); }
  else { return std::floor(x); }
}
\end{lstlisting}
\end{block}

\end{itemize}
\end{frame}

\begin{frame}[t,fragile]{Enforcemente}
\begin{itemize}
  \item Flag functions with control flow based on the value of variables
        declared at namespace scope.
  \item Flag functions that modify the value of variables at namespace scope.
\end{itemize}
\end{frame}

\subsection{I5: Express preconditions}

\begin{frame}[t,fragile]{Expressing preconditions}
\begin{itemize}
  \item \textmark{Precondition}: Conditions that must be meet 
        for a function to work properly.
  \item Preconditions can be expressed at function entry.
    \begin{itemize}
      \item You can use a library like GSL.
      \item \url{https://github.com/Microsoft/GSL}
    \end{itemize}
\end{itemize}

\mode<presentation>{\vfill\pause}
\begin{block}{Good}
\begin{lstlisting}
double sqrt(double x) {
  Expects(x>=0);
  // ...
}
\end{lstlisting}
\end{block}

\mode<presentation>{\vfill\pause}
\begin{itemize}
  \item Enforcing: Cannot be enforced.
\end{itemize}

\end{frame}


\begin{frame}[t]{Interfaces guidelines (II)}
\begin{itemize}
  \item More guidelines:
  \begin{itemize}
    \item I.11: Never transfer ownership by a raw pointer (\cppid{T*}) or reference (\cppid{T\&})
    \item I.12: Declare a pointer that must not be null as \cppid{not\_null}
    \item I.13: Do not pass an array as a single pointer
    \item I.22: Avoid complex initialization of global objects
    \item I.23: Keep the number of function arguments low
    \item I.24: Avoid adjacent parameters that can be invoked by the same arguments in either order with different meaning
    \item I.25: Prefer empty abstract classes as interfaces to class hierarchies
    \item I.26: If you want a cross-compiler ABI, use a C-style subset
    \item I.27: For stable library ABI, consider the Pimpl idiom
    \item I.30: Encapsulate rule violations
  \end{itemize}
\end{itemize}
\end{frame}

\subsection{I11: Never transfer ownership by raw pointer or reference}

\begin{frame}[t,fragile]{Resource ownership and pointers}
\begin{itemize}
  \item Returning a pointer to an allocated resource makes unclear who should
        deallocate.

\mode<presentation>{\vfill\pause}
\begin{block}{Bad}
\begin{lstlisting}
image * load_image(std::string_view file_name) {
  image * p_image = new image;
  //...
  return p_image;
}

void f() {
  image * img = load_image("map.jpg");
  //...
  // Who's responsible for delete?
}
\end{lstlisting}
\end{block}

\end{itemize}
\end{frame}

\begin{frame}[t,fragile]{Resource ownership and references}
\begin{itemize}
  \item Returning references make things harder.

\mode<presentation>{\vfill\pause}
\begin{block}{Bad}
\begin{lstlisting}
image & load_image(std::string_view file_name) {
  image * p_image = new image;
  //...
  return *p_image;
}

void f() {
  image & img = load_image("map.jpg");
  //...
  // Who's responsible for delete?
}
\end{lstlisting}
\end{block}
\end{itemize}
\end{frame}

\begin{frame}[t,fragile]{Returning resources with smart pointers}
\begin{itemize}
  \item A smart pointer is a resource manager for memory.

\mode<presentation>{\vfill\pause}
\begin{block}{Good}
\begin{lstlisting}
std::unique_ptr<image>(std::string_view file_name) {
  auto new_image = std::make_unique<image>(); // Allocates an image
  // ...
  return new_image;
}

void f() {
  auto img = load_image("map.jpg");
  //...
} // Automatically deallocate img
\end{lstlisting}
\end{block}
\end{itemize}
\end{frame}

\begin{frame}[t,fragile]{Returning resources by value}
\begin{itemize}
  \item If the type supports move semantics, the return by value.

\mode<presentation>{\vfill\pause}
\begin{block}{Good}
\begin{lstlisting}
image load_image(std::string_view file_name) {
  image new_image;
  // ...
  return new_image;
}

void f() {
  auto img = load_image("map.jpg");
  //...
} 
\end{lstlisting}
\end{block}
\end{itemize}
\end{frame}

