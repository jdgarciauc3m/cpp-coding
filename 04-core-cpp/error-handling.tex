\section{E: Error Handling}

\begin{frame}[t]{Error handling}
\begin{itemize}
  \item Error handling involves:
    \begin{itemize}
      \item Detecting an error
      \item Transmitting information about an error to some handler code
      \item Preserving a valid state of the program
      \item Avoiding resource leaks
    \end{itemize}

  \mode<presentation>{\vfill\pause}
  \item Goals of guidelines:
    \begin{itemize}
      \item Type violations (e.g., misuse of unions and casts)
      \item Resource leaks (including memory leaks)
      \item Bounds errors
      \item Lifetime errors (e.g., accessing an object after is has been deleted)
      \item Complexity errors (logical errors made likely by overly complex expression of ideas)
      \item Interface errors (e.g., an unexpected value is passed through an interface)
    \end{itemize}
\end{itemize}
\end{frame}

\begin{frame}[t]{Error handling}
\begin{itemize}
  \item Guidelines:
  \begin{itemize}
      \item E.1: Develop an error-handling strategy early in a design
      \item E.2: Throw an exception to signal that a function can’t perform its assigned task
      \item E.3: Use exceptions for error handling only
      \item E.4: Design your error-handling strategy around invariants
      \item E.5: Let a constructor establish an invariant, and throw if it cannot
      \item E.6: Use RAII to prevent leaks
      \item E.7: State your preconditions
      \item E.8: State your postconditions
  \end{itemize}
\end{itemize}
\end{frame}

\begin{frame}[t]{Error handling}
\begin{itemize}
  \item Guidelines:
  \begin{itemize}
      \item E.12: Use noexcept when exiting a function because of a throw is impossible or unacceptable
      \item E.13: Never throw while being the direct owner of an object
      \item E.14: Use purpose-designed user-defined types as exceptions (not built-in types)
      \item E.15: Throw by value, catch exceptions from a hierarchy by reference
      \item E.16: Destructors, deallocation, and swap must never fail
      \item E.17: Don’t try to catch every exception in every function
      \item E.18: Minimize the use of explicit try/catch
      \item E.19: Use a \cppid{final\_action} object to express cleanup if no suitable resource handle is available
  \end{itemize}
\end{itemize}
\end{frame}

\begin{frame}[t]{Error handling}
\begin{itemize}
  \item Guidelines:
  \begin{itemize}
      \item E.25: If you can’t throw exceptions, simulate RAII for resource management
      \item E.26: If you can’t throw exceptions, consider failing fast
      \item E.27: If you can’t throw exceptions, use error codes systematically
      \item E.28: Avoid error handling based on global state (e.g. \cppid{errno})
      \item E.30: Don’t use exception specifications
      \item E.31: Properly order your catch-clauses
  \end{itemize}
\end{itemize}
\end{frame}
